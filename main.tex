\documentclass[a4paper,12pt]{article}
\usepackage{hyperref}

\setlength\parindent{0pt}

\title{CS 224 Object Oriented Programming and Design Methodologies}
\author{Assignment 04}
\date{Out: 5 October, Due: 15 Oct, 1830h}

\begin{document}
\maketitle

\section{Task}

For this assignment you will be creating a drawing program of your own. You are provided with a folder \path{Artistik} that contains sample code. The code uses {\tt SDL 2} about which you can learn through the \href{\href{http://lazyfoo.net/tutorials/SDL/}}{lazyfoo tutorials}. The code provides two structures, {\tt Point} and {\tt Color}, and allows to draw a red rectangle by left clicking the mouse and dragging. Your task is to extend the sample code as defined below.

When the sample code is run, each drawn rectangle causes the previous rectangle to be lost. You are required to store all drawn rectangles in a stack. Every drawn rectangle should be stored in the stack. As rectangles are drawn on the screen, the stack keeps getting populated. You need to implement the following.

\begin{itemize}
\item You will need to declare a {\tt Shape} class and derive the classes {\tt Rect} and {\tt Line} from it.
\item You will create a {\tt Stack} class that implements a stack containing pointers to {\tt Shape} objects.
\item You will need to study {\tt SDL}'s documentation to understand how a line is drawn.
\item Use a {\tt Color} object to assign a random color to every newly drawn line or rectangle.
\item The user should use they keys {\tt r} and {\tt l} to select the drawing of a {\tt Line} or {\tt Rectangle} respectively.
\item Clicking the right mouse button will undo the last drawn shape and pop it from the stack.
\item Popped shapes will be stored in a separate {\it undo stack}.
\item Clicking the middle mouse button will {\it redo}--it will pop the topmost shape from the undo stack and push it on the first stack.
\item When a new shape is drawn after an undo operation, the undo stack will be purged.
\item {\tt -} and {\tt +} keys will change the order of the last drawn object within the stack. {\tt -} will cause the shape to go deeper in the stack and {\tt +} will move the shape toward the top.
\item There is a memory leak in the {\tt main} program. You are required to identify it by commenting it appropraitely and to fix it.
\end{itemize}

\begin{center}
  -- The End --
\end{center}


\end{document}